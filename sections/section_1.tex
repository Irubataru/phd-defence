\section{Introduction}

\begin{frame}{Forces of nature}

  \begin{center}
    \includegraphics[width=\textwidth]{force_grid}
  \end{center}
  \vspace*{-.75\baselineskip}

  \begin{overprint}
    \onslide<1>
      The \alert{strong force} is non-Abelian, has self interactions and is
      non-perturbative at low energies
    \onslide<2>
      QCD can be treated perturbatively at \alert{high energies} due to the \alert{running} of
      the gauge coupling
  \end{overprint}

\end{frame}

\begin{frame}{Putative QCD phase diagram}

  \begin{center}
    \includegraphics[width=0.85\textwidth]{qcd_phase_diag}
  \end{center}

  \vspace{-1em}

  Both experimental and theoretical endeavours can probe different regions of the
  phase diagram in different ways
  
\end{frame}

\begin{frame}{Lattice regularisation}

  We regularise the quantum field theory by introducing a smallest scale, the
  \alert{lattice spacing} (\alert{$a$})

  \vspace{.75em}
  
  \tikz \node[draw=ColourDark!50, fill=ColourDark!15, minimum height=1cm] {%
    \begin{minipage}{\textwidth}\vspace*{-1ex}
      \[
        (x_0, x_1, x_2, x_3) \in \mathbb{R}^4
        \:\tikz[baseline=-.5ex] \draw[-{Stealth[round]}] (0,0) -- (4mm,0);\:
        (n_0, n_1, n_2, n_3) \in \scalemath{0.85}{\big\{0, 1, \dots, N-1\big\}^4}
      \]
    \end{minipage}};

  \vspace{.75em}

  The {\color{ColourHl1}quark fields} are placed on the {\color{ColourHl1}lattice sites}, while the {\color{ColourBase}gauge fields} are
  placed on the {\color{ColourBase}links} connecting them

  \begin{center}
    \begin{tikzpicture}[remember picture]
      \draw[thin, ColourDark!50] (-0.25, -0.25) grid (6.25, 1.25);

      \draw[link] (1,0) -- (1,1);
      \node[fermion] at (1,0) {};
      \node[fermion] at (1,1) {};
      \coordinate (propagator-bottom) at (1,-.25);

      \draw[link] (2,1) -- (3,1) -- (3,0);
      \node[fermion] at (2,1) {};
      \node[fermion] at (3,0) {};

      \draw[plaquette] (4,0) -- ++(1,0) -- ++(0,1) -- ++(-1,0) -- cycle;
      \coordinate (plaquette-mid) at (4.5,-.25);
    \end{tikzpicture}
  \end{center}

  \nointerlineskip
  \begin{tikzpicture}[overlay,remember picture]
      \draw[ColourDark,{Stealth[round]}-,thick] ([yshift=-1pt]propagator-bottom) ..
        controls +(0,-.5cm) and +(.5cm, 0) .. ++(-.75cm, -.5cm)
        node [left, scale=0.65, align=center] {Covariant\\derivative term};
      \draw[ColourDark,{Stealth[round]}-,thick] ([yshift=-1pt]plaquette-mid) ..
        controls +(0,-.5cm) and +(.5cm, 0) .. ++(-.75cm, -.5cm)
        node [left, scale=0.65, align=center] (desc) {Gauge\\force term};
      \node[anchor=north, scale=0.5] at (desc.south) {(also known as the
        plaquette)};
  \end{tikzpicture}
\end{frame}

\begin{frame}{Lattice simulations at \texorpdfstring{$\mu > 0$}{m > 0}: The sign problem}

  The partition function is given by the gauge integral

  \tikz \node[draw=ColourDark!50, fill=ColourDark!15, minimum height=1.25cm] {%
    \begin{minipage}{\textwidth}
  \[
    \mathcal{Z} = \int \hspace*{-3pt} \mathrm{d} U_{\nu}\scalemath{0.85}{(n)} \: \det Q (\mu_B) \exp (- S_g )
  \]
    \end{minipage}};

  Numerical evaluation relies on \alert{importance sampling} using\\
  the integrand as a probability distribution

  \vspace{.5em}

  \tikz \node[draw=ColourHl1!75, fill=ColourHl1!25, minimum height=1cm] {%
    \begin{minipage}{\textwidth}\vspace{-.5\abovedisplayskip}
  \[
    P(\mathrm{conf}) \propto \det Q (\mathrm{conf}, \mu_B) \exp (-
    S_g(\mathrm{conf}))
  \]
    \end{minipage}};
  
  \vspace{.5em}

  However, for $\real(\mu_B) \neq 0: \alert{\det Q(\mu_B) \in \mathbb{C}}$, but
  $\mathcal{Z}(\mu_B) \in \mathbb{R}$

  \vspace{.5em}

  This is not just a numerical obstacle, but a \alert{fundamental issue}
  
\end{frame}

\begin{frame}{Lattice simulations at \texorpdfstring{$\mu > 0$}{m > 0}: The sign problem (sketch)}

  \begin{center}
    \includegraphics[width=.8\textwidth]{oscillating_sign}
  \end{center}

  \vspace{.5em}

  The sign problem is one of the \alert{biggest obstacles} in obtaining first
  principle calculations of dense systems

\end{frame}

\begin{frame}{Lattice simulations at \texorpdfstring{$\mu > 0$}{m > 0}: Lattice saturation (sketch)}

  \begin{center}
    \includegraphics[width=.8\textwidth]{saturation_sketch}
  \end{center}

  \vspace{-.2cm}

  As $\mu_B$ increases, the available lattice points fills up, and the
  \alert{Pauli exclusion principle} halts any further dynamics
  
\end{frame}

\begin{frame}{Lattice simulations at \texorpdfstring{$\mu > 0$}{m > 0}: Lattice saturation}

  \begin{center}
    \includegraphics[width=.8\textwidth]{saturation_plot}
  \end{center}

  \vspace{-.2cm}

  Very \alert{fine lattices} and \alert{continuum extrapolations} are needed to
  access dense physics on the lattice

  
\end{frame}
