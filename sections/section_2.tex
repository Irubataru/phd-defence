\section{The Effective Theory}

\begin{frame}{The general concept}

  Lessen the sign problem by carrying out some of the gauge integrals
  analytically

  \vspace{.25em}

  \tikz \node[draw=ColourHl1!75, fill=ColourHl1!25, minimum height=2.35cm] {%
    \begin{minipage}{\textwidth}\vspace*{-2ex}
      \begin{align*}
        \mathcal{Z} &= \int \mathrm{d} U_{\nu}\: \exp\big\{ \scalemath{0.85}{-} S_{\mathrm{action}} \big\} \\
        &= \int \mathrm{d} U_0\: \exp\big\{ \scalemath{0.85}{-} S_{\mathrm{effective \: action}} \big\}
      \end{align*}
    \end{minipage}};

  $S_{eff}$ cannot be computed analytically, we therefore expand around two limits

  \begin{itemize}
    \item {\color{ColourHl1}\makebox[3.1cm][l]{Heavy quarks:}
        $m_q \:\tikz[baseline=-.5ex] \draw[-{Stealth[round]}] (0,0) -- (4mm,0);\: \infty$ \hfill
        \raisebox{.1ex}{\scalebox{0.75}{(hopping parameter expansion)}}}
    \item {\color{ColourBase}\makebox[3.1cm][l]{Strong coupling:}
        $g \:\tikz[baseline=-.5ex] \draw[-{Stealth[round]}] (0,0) -- (4mm,0);\: \infty$ \hfill
        \raisebox{.1ex}{\scalebox{0.75}{(character expansion)}}}
  \end{itemize}
  
\end{frame}

\begin{frame}{Pictorial description}

  \vspace{.25cm}
  \begin{center}
    \begin{tikzpicture}[remember picture]
      \node                       (stage1) {\includegraphics[scale=0.95]{pure_gauge_strip_stage1}};
      \node[below=.5cm of stage1] (stage2) {\includegraphics[scale=0.95]{pure_gauge_strip_stage2}};
      \node[below=.5cm of stage2] (stage3) {\includegraphics[scale=0.95]{pure_gauge_strip_stage3}};
    \end{tikzpicture}
  \end{center}

  \nointerlineskip
  \begin{tikzpicture}[overlay,remember picture]

    % Axes
    % ----------------------------------------------------------------------------------------------------------
    \draw[thin, ColourDark, -{Stealth[round, length=3pt]}] ([shift={(225:.1cm)}] stage1.south west) -- +(0, 1cm)
      node[above,scale=0.5] {$x$};
    \draw[thin, ColourDark, -{Stealth[round, length=3pt]}] ([shift={(225:.1cm)}] stage1.south west) -- +(1cm, 0)
      node[right,scale=0.5] {$t$};

    \draw[thin, ColourDark, -{Stealth[round, length=3pt]}] ([shift={(225:.1cm)}] stage2.south west) -- +(0, 1cm)
      node[above,scale=0.5] {$x$};
    \draw[thin, ColourDark, -{Stealth[round, length=3pt]}] ([shift={(225:.1cm)}] stage2.south west) -- +(1cm, 0)
      node[right,scale=0.5] {$t$};

    \draw[thin, ColourDark, -{Stealth[round, length=3pt]}] ([shift={(225:.1cm)}] stage3.south west) -- +(0, 1cm)
      node[above,scale=0.5] {$y$};
    \draw[thin, ColourDark, -{Stealth[round, length=3pt]}] ([shift={(225:.1cm)}] stage3.south west) -- +(1cm, 0)
      node[right,scale=0.5] {$x$};
    % ----------------------------------------------------------------------------------------------------------

    \draw[thick, ColourDark, -{Stealth[round]}] ([xshift=.2cm] stage1.east)
      .. controls +(-25:1cm) and +(25:1cm) .. ([xshift=.2cm] stage2.east)
      node[midway, right=.2cm, text width=2cm, scale=0.65]
        {Integrating out spatial {\color{ColourBase}gauge links}\rlap{...}};

    \draw[thick, ColourDark, -{Stealth[round]}] ([xshift=-.4cm] stage2.west)
      .. controls +(205:1cm) and +(155:1cm) .. ([xshift=-.4cm] stage3.west -|
      stage2.west)
      node[yshift=.1cm, anchor=east, midway, left=.2cm, text width=2.75cm, align=right, scale=0.65]
        {gives effective interactions\\between {\color{ColourBase}Polyakov loops}};

    \node[right=.5cm of stage3] [scale=0.65, text width=3.5cm] {%
      Similar treatment of the {\color{ColourHl1}quark\\contribution}};

  \end{tikzpicture}
\end{frame}

\begin{frame}{The effective lattice theory}

  After integration we have a \alert{Polyakov loop effective theory}

  \vspace{.5em}

  \tikz \node[draw=ColourHl1!75, fill=ColourHl1!25, minimum height=2.25cm] {%
    \scalebox{0.9}{
    \begin{minipage}{1.1\textwidth}
    \[
      S_{\mathrm{eff}}
        \sim \lambda_1 \sum_{\langle x, y \rangle} L(x) L^*(y)
        + h_2\sum_{\langle x, y \rangle} W[L](x) W[L](y) + \dots
    \]
    \end{minipage}}};

  %\vspace{1em}

  The theory is contained in the set of coupling constants

  \vspace{.5em}

  \tikz \node[draw=ColourDark!50, fill=ColourDark!15, minimum height=1cm] {%
    \begin{minipage}{\textwidth}\vspace*{-1ex}
    \[
      \lambda_1(\beta, N_t, \kappa), \lambda_2, ... \hspace{.5cm} h_1(\beta,N_t,\kappa), h_2, ...
    \]
    \end{minipage}};

  The sign problem of which is weak enough to use\\
  \alert{complex Langevin} or \alert{reweighting} to evaluate numerically
  
\end{frame}
